\documentclass[a4paper]{article}

\usepackage[top=3cm, bottom=2cm, left=2cm, right=2cm]{geometry}
\usepackage{graphicx}
\usepackage{fancyhdr}
\usepackage{scrextend}
\usepackage[hidelinks]{hyperref}
\usepackage{textcomp}
\usepackage{tipa}

\graphicspath{ {./images/} }

\title{Homework 2 - Machine Learning: Weather Classification}
\author{Edoardo Piroli - 1711234}

\pagestyle{myheadings}
\pagestyle{fancy}
\fancyhf{}

\renewcommand{\headrulewidth}{0pt}
\renewcommand{\footrulewidth}{1pt}

\fancyfoot[C]{Report HW2 - Machine Learning}
\fancyfoot[R]{\thepage}

\begin{document}

\maketitle
\thispagestyle{empty}

\newpage
\tableofcontents
\thispagestyle{empty}
\newpage

\pagenumbering{arabic}

\section{Introduction}
\subsection{Assignment}
The assignment of the homework was to build an image classifier capable of assigning pictures to one of the following classes: \{\textit{RAINY}, \textit{HAZE}, \textit{SUNNY}, \textit{SNOWY}\}; based on how the weather was like when the picture was taken. In particular, the request was to address the former problem in 2 different ways: (1) defining and training a CNN from scratch for this particular task and (2) applying transfer learning and fine-tuning on a pre-trained model.
\subsection{Dataset}
The provided datasets\footnote{available here: \href{https://drive.google.com/drive/folders/1UzH28Q8xki8_DMYdDgHxi40-CJ800Kaq}{https://drive.google.com/drive/folders/1UzH28Q8xki8\textunderscore DMYdDgHxi40-CJ800Kaq}} are separated in several different archives and folders. I have chosen to use \textit{MWI-Dataset-1.1\textunderscore 2000} as my training set and \textbf{\textit{ADD HERE}} as my testing set. The former contains 500 different pictures per class, while the latter \textbf{CONTINUE HERE}.
\end{document}